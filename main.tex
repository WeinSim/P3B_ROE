\documentclass{article}

\usepackage[utf8]{inputenc}
\usepackage[german]{babel}
\usepackage[
    a4paper, top=2.5cm, bottom=2.5cm, left=2.5cm, right=2.5cm, marginparwidth=1.75cm
]{geometry}
\usepackage{amsmath}
\usepackage{amsfonts}
\usepackage{enumitem}
\usepackage[
    colorlinks=true, 
    citecolor=black,
    filecolor=black,
    linkcolor=black,
    urlcolor=blue
]{hyperref}
\usepackage{graphicx}
\usepackage{amssymb}
\usepackage{float}
\usepackage{pdfpages}
\usepackage{multirow}
\usepackage[
    tocskip=0.1\baselineskip, skip=0.7\baselineskip, parfill
]{parskip}
\usepackage{listings}
\usepackage{fancyhdr}
\usepackage{xcolor}

% Kopf- und Fußzeile
\pagestyle{fancy}
\fancyhf{}
%Kopfzeile mittig mit Kaptilname
\fancyhead[C]{\nouppercase{\leftmark}}
%Fußzeile links bzw. innen
\fancyfoot[L]{\versuchsname}
%Fußzeile mittig (Seitennummer)
\fancyfoot[R]{\thepage}
\renewcommand{\footrulewidth}{0.35pt}

% Hilfs-Commands für Gleichungen
\newcommand{\widespace}{\enspace}
\newcommand{\wideeq}{\widespace = \widespace}
\newcommand{\wideneq}{\widespace \neq \widespace}
\newcommand{\wideapprox}{\widespace \approx \widespace}
\newcommand{\wideleq}{\widespace \leq \widespace}
\newcommand{\widegeq}{\widespace \geq \widespace}
\newcommand{\widele}{\widespace \le \widespace}
\newcommand{\widege}{\widespace \ge \widespace}
\newcommand{\wideiff}{\widespace \iff \widespace}
\newcommand{\wideimplies}{\widespace \implies \widespace}
\newcommand{\pd}[2]{
    \frac{\partial #1}{\partial #2}
}
\newcommand{\result}[2]{
    #1 \, \text{#2}
}

% Markieren von verwendetem Code
\definecolor{codebg}{RGB}{230, 240, 255}
\newcommand{\coderef}[1]{
    \text{
        \enspace
        \footnotesize
        \colorbox{codebg}{\texttt{#1()}}
    }
}

% Formatieren von Code im Anhang
\definecolor{codegreen}{rgb}{0,0.6,0}
\definecolor{codegray}{rgb}{0.5,0.5,0.5}
\definecolor{codepurple}{rgb}{0.58,0,0.82}
\definecolor{backcolour}{rgb}{0.95,0.95,0.92}
\lstdefinestyle{mystyle}{
    backgroundcolor=\color{backcolour},   
    commentstyle=\color{codegreen},
    keywordstyle=\color{magenta},
    numberstyle=\tiny\color{codegray},
    stringstyle=\color{codepurple},
    basicstyle=\ttfamily\footnotesize,
    breakatwhitespace=false,         
    breaklines=true,                 
    captionpos=b,                    
    keepspaces=true,                 
    numbers=left,                    
    numbersep=5pt,                  
    showspaces=false,                
    showstringspaces=false,
    showtabs=false,                  
    tabsize=4
}
\lstset{style=mystyle}

% Formatierung von Absätzen
\renewcommand{\baselinestretch}{1.2}

\allowdisplaybreaks

% Allgemeine Infos
\newcommand{\versuchsname}{
    ROE - Röntgenstrahlung: Braggreflexion und Röntgenfluoreszenzanalyse
}
\newcommand{\githuburl}{
    \url{https://github.com/WeinSim/P3B/FHV}
}

% Titel und Autor
\title{\versuchsname}
\author{Simon Weinzierl, Yannic Werner}

\begin{document}

\maketitle

\begin{center}
    Physikalisches Fortgeschrittenenpraktikum P3B
    nach der Studienordnung für Studienbeginn bis WS 2022/23
\end{center}

\vspace*{6cm}

\begin{center}
    \footnotesize
    Alle Teile dieses Dokuments (Vorbereitung, Protokoll, Auswertung) wurden
    von beiden Teilnehmern in gleichen Teilen und ohne fremde Hilfe bearbeitet.
    Sofern fremde Quellen verwendet wurden, sind diese angegeben.

    Der \LaTeX-Code ist auf GitHub unter \githuburl verfügbar.
    
    © Alle Rechte vorbehalten.
\end{center}

 % LMU-Siegel
\AddToShipoutPicture*{
    \put(315,0){
        \parbox[b][5cm]{5cm}{
            \includegraphics[width=10cm]{Abbildungen/Siegel_LMU.pdf}
        }
    }
}

\newpage

% Inhaltsverzeichnis
\tableofcontents

\newpage

% Literatur
\bibliographystyle{alpha}
\bibliography{literatur}

\newpage

\section{Vobereitung}

\subsection{Physikalischer Hintergrund}

    \subsubsection{Röntgenröhre}
    
    \cite{medizinphysik}
    
    Röntgenstrahlung entsteht durch die Geschewindigkeitsänderung geladener Teilchen. Dabei wird die Röntgen-strahlung durch das Abbremsen energiereicher Elektronen in einer Röntgenröhre erzeugt.
    
    Folgend ist der Aufbau einer Röntgenröhre skizzenhaft dargestellt:

    \begin{figure}[H]
        \centering
        \includegraphics[width=0.7\linewidth]{Abbildungen/Röntgenröhre.pdf}
        \caption{Skizzenhafte Röntgenröhre}
    \end{figure}

    Durch das anlegen einer Heizspannung von einigen Volt an der Kathode, treten, bedingt durch den thermoelektrischen Effekt, schwach, oder gar nicht gebundene Elektronen aus der Heizwendel aus. Durch die angelegte Anodenspannung werden die Elktronen von der Kathode zur Anode beschleunigt und treffen anschließend auf die Anode.

    Durch die Bremsstrahlung und durch die Wechselwirkungen untern den einzelnen Elektronen kommt es zur Entstehung von Röntgenstrahlung. Deren Überlagerung bildt das emittierte Röntgenspektrum. Die Eigenschaften und die Entstehung von Bremsstrahlung und der charakteristischen Strahlung werden im Kapitel 1.1.3 genauer beleuchtet. 

    \subsubsection{Linienspektrum eines Stoffe, Schalenmodell, mögliche Übergänge (Quantenmechanik)}

    \cite{chemie.de}
    \cite[1050--1051]{Physik}
    \cite{studyfix}

    Ein Linienspektrum ist ein Strahlungsspektrum, welches voneinander getrennte, diskrete Linien zeigt. Das können zum Beispiel Absoroptions- oder Emissionlinien in Lichtspektren sein. Jedoch weisen auch manche Teilchenstrahlungen, wie die Alphastrahlung Lininespektren auf. Daraus kann man ableiten, dass auch Teilchen diskrete kinetische Energien haben.

    Jedes Material (Atom, Molekül) hat dabei chrakteristische, diskrete Energieniveaus, auf welchen sich die Elektronen befinden können. Der Wechsel von einem Energienievau auf ein anderes erfolgt dabei durch Aufnahme/Abgabe eines Pohotons. Aus der Energiedifferenz lässt sich dann die zugehörige Wellenlänge über folgende Formel bestimmen: $\lambda=\frac{c}{v}$.

    Ein angeregtes Atom oder Molekül befindet sich immer nur sehr kurz in seinem angeregten Zustand. Es fällt nach einer ehr kurzen Zeitspanne wieder in einen tieferen Energiezustand zurück. Die ausgesandten Photonen erscheinen dann mit einer ganz bestimmten Energie als Emissionslienien auf einem Spektrum. So entstehen chrakteristische Spektren, wie zum Beispiel das Spektrum von Wasserstoff oder Helium.


    Das Schalenmodell ist ein Modell, um den Aufbau von Atomen zu beschreiben. Das Schalenmodell basiert hierbei auf dem Bohr'schen Atommodell. Im Atomkern befinden sich die Protonen und Neutronen. Die negativ geladenen Elektronen bewegen sich in Schalen um den positiv gelandenen Kern. Folgende Skizze soll das Schalenmodll verdeutlichen:

    \begin{figure}[H]
        \centering
        \includegraphics[width=0.3\linewidth]{Abbildungen/Schalenmodell.pdf}
        \caption{Schalenmodell}
    \end{figure}

    Die Schalen haben, wie in der Abbildung ersichtlich, unterschiedlich Abstände zum Kern. Die Schalen werden von innen nach außen mit unterschiedlichen Buchstaben bezeichnet, beginnend mit der "K"-Schalge, "L"-Schale und "M"-Schale. Dabei haben die Schalen unterschiedlich viel Platz für Elektronen  Über die Formel $e=2n^2$ lässt sich die maximale Anzahl an Elektronen für jede Schale bestimmen.

    Jedoch hat das Schalenmodell auch Grenzen. Will man Elemente mir mehr als 20 Elektronen beschreiben, so ist das Schalenmodell zur Beschreibung nicht mehr geeignet!


    Der Zusammenhang mit der Quantenmechanik ist nun leicht hergestellt. Die Quantenmechanik besagt, dass ein Elektron in einem Atom nur auf bestimmen Energieniveaus existieren kann und nicht zwischen den Niveaus! Wie oben bereits beschrieben können also folgende Aussagen getroffen werden:
        \begin{enumerate}[label=\arabic*.]
            \item Abstieg: ein Photon wird emittiert (von $E_2$ nach $E_1$)
            \item Aufstieg: ein Photon wird absorbiert
        \end{enumerate}
    Die Energie des Photons wird also genau durch den Energieunterschied der beiden Zustände beschrieben: 
    \begin{center}
        $E_{Photon}=E_{oben}-E_{unten}=h*f=\frac{h*c}{\lambda}$ 
    \end{center}

    mit $h$ als Planksches Wirkungsquantum, $f$ als Frequenz, $\lambda$ als Wellenlänge und $c$ als Lichtgeschwindigkeit.

    Folgende Aussagen lassen sich also zusammenfassend treffen:
    \begin{enumerate}[label=\arabic*.]
            \item Die Energiedifferenzen werden durch Linienspektren beschrieben
            \item Die Energiedifferenzen entstehen durch Übergänge von Elektronen zwischen den unterschiedglichen Energieniveaus
            \item Jedem Übergang ist eine chrakteristische Linie zugeordnet
            \item Jedes Atom wird durch ein charakteristisches Linienspektrum beschrieben
        \end{enumerate}

    Beispiel Wasserstoffatom:
    \begin{itemize}
        \item Lyman-Serie: Übergang nach $n=1$
        \item Balmer-Serie: Übergang nach $n=2$
        \item Paschen-Serie: Übergang nach $n=3$
        \item Bracket-Serie: Übergang nach $n=4$
        \item Pfund-Serie: Übergang nach $n=5$
    \end{itemize}
    Hier ist wohl die Balmer-Serie die bekannteste Serie, da hier das Linienspektrum mit dem bloßen Auge sichtbar ist.

    \subsubsection{Röntgenstrahlung (Bremsstrahlung, chrakteristische Strahlung)}

    \cite{Physik} [1010--1011; 1384]
    \cite{leifiphysik_B}
    \cite{leifiphysik_C}

    Wie bereits in Kapitel 1.1.1 erwähnt, wird in diesem Kapitel die Bremsstrahlung und die charakteristische Strahlung genauer betrachtet.
    
    Die Kathode in der Röntgenröhre beschleunigt die Elektronen auf eine Geschewindigkeit von circa $0,35c$. Die ausgetretenen Elektronen treffen nun mit dieser hohen Geschwindigkeit auf die Anode in der Röntgenröhre. Durch das extreme Abbremsen der Elektronen, entsteht elektrische Strahlung, die sogenannte Bremsstrahlung. Je stärker die Elektronen abgebremst werden, umso mehr Strahlung senden diese aus. Da nicht alle Elektronen gleich stark abgebremst werden, besitzen die Photonen der Bremsstrahlung unterschiedliche Wellenlängen. Dadurch ist das Spektrum der Bremsstrahlung ein kontinuierliches Spektrum. Die minimale Wellenlänge wird häufig als $\lambda{gr}$ bezeichnet. Die minimale Wellenlänge lässt sich durch folgende Formel bestimmen:
    \begin{center}
        $\lambda_{gr}=\frac{h•c}{e•U}$
    \end{center}
    wobei $h$ das Planksche Wirkungsquantum, $c$ die Lichtgeschwindigkeit, $e$ die Elementarladung und $U$ die Beschleunigungsspannung ist.

    Aus der Formel lässt sich ableiten, dass die minimale Wellenlänge und die Beschleunigungsspannung indirekt proportional zueinander sind. Mit steigender Beschleunigungsspannung sinkt die minimale Wellenlänge.

    Abschließend lässt sich also zur Bremsstrahlung folgendes zusammenfassen: das kontinuierliche Spektrum entsteht durch das Abbremsen der ausgesandeten, beschleunigten Elektronen der Kathode. Durch das unterschiedlich starke Abbremsen der Elektronen werden Photonen mit unterschiedlichen Energiewerten entsandt, wobei die minimale Wellenlänge/die maximale Photonenenergie von der Beschleunigungsspannung der Röntgenröhre abhängen.

    Neben der Bremsstrahlung, tritt bei ausreichend großer Beschleunigungsspannung auch ein charakteristisches Linienspektrum auf. Diese Röntgenstralhung is jedoch nur bei höheren Ordnungszahlen auf. Das liegt daran, dass Atome mit höherer Ordnungszahl zahlreiche Atome in den äußeren Elektronenschalen haben. Diese sind nicht so stark an den Atomkern gebunden und können somit leichter austreten als Elektronen, welche sich nahe dem Kern befinden. Im folgenden soll die Entstehung der chrakteristischen Stralung stichpunktartig beschrieben werden:
    \begin{enumerate}
        \item Ein Anode-Atom wird durch ein sehr schnelles Elektron angeregt. Ein Elektron wird folgich auch ein noch freies, höheres, Energieniveau angehoben. Auf einer der unteren Schalen entsteht somit eine Lücke. Wie bekannt, gehen angeregt Atome nach kurzer Zeit in energetisch günstigere Zustände über. Folglich kann folgendes passieren:
        \item Möglichkeit 1: Ein Elektron der höchsten Schale fällt nun wieder zurück auf die Lücke der untersten Schale. Dabei wird ein Photon emittiert.
        \item Möglichkeit 2: Die Lücke der untersten Schale wird durch ein Elektron auf einer höheren Schale aufgefüllt. Hierbei wird ein Photon emittiert. Die so enstandene Lücke in der höheren Schale wird erneut durch ein Elektron einer Schale darüber aufgefüllt. Es wird wieder ein Photon emittiert. Dieser Vorgang läuft so lange ab, bis wieder alle Schalen aufgefüllt sind, bis auf die äußerste Schale.
    \end{enumerate}

    Zusammenfassend lässt sich also sagen, dass das Röntgen-Spektrum durch charakteristische Strahlung identifiziert werden kann. Elektronenübergänge zwischen den unterschiedlichen Schalen der Atome sind die Ursache für das Auftreten der charakteristischen Strahlung.

    \subsubsection{Zustandekommen der charakteristischen Strahlung im Röntgenspektrum ($K_\alpha$, $K_\beta$)}

    In Kapitel davor schon beschrieben. Schauen, wie man das in diesem Kapitel noch einbauen kann. Evtl Skizze und Verweis nach oben???? Übernahme der Stichpunkte nach unten?!
    
\newpage

\subsection{Aufgaben aus dem Text}

    \subsubsection{Teile der jeweiligen Aufgaben}
[...]


\newpage

\section{Veruschsablaufplan}

\begin{center}

\end{center}

\subsection{Benötigte Materialien}
    \begin{enumerate}[label=\arabic*.]
        \item Eins
        \item Zwei
    \end{enumerate}

\newpage

\subsection{Teilversuch 1: Bragg-Reflexion von Röntgenstrahlung des Molybdän an einem NaCl-Eiskristall}
\begin{enumerate}[label = (\Roman*)]
    \item Ziel: ...
    
    \item Versuchsmethode: ...
    
    \item Versuchsskizze:
    
        \begin{figure}[H]
        \centering
        \includegraphics[width=0.7\linewidth]{Bild}
        \caption{Versuchsskizze Teilversuch 1}
        \end{figure}

    \item Planung der Durchführung
        \begin{itemize}
            \item eins
            \item zwei
        \end{itemize}

    \item Vorüberlegungen zur Durchführung \& Auswertung
        \begin{itemize}
            \item eins
            \item zwei
        \end{itemize}
    
\end{enumerate}

\newpage

\subsection{Teilversuch 2: Energiespektrum einer Röntgenröhre in Abhängigkeit der Spannung}
\begin{enumerate}[label = (\Roman*)]
    \item Ziel: 
    
    \item Versuchsmethode: 
    
    \item Versuchsskizze:
    
        \begin{figure}[H]
        \centering
        \includegraphics[width=0.7\linewidth]{Bild}
        \caption{Versuchsskizze Teilversuch 2}
        \end{figure}

    \item Planung der Durchführung
        \begin{itemize}
           \item eins
           \item zwei
        \end{itemize}

    \item Vorüberlegungen zur Durchführung \& Auswertung
        \begin{itemize}
            \item eins
            \item zwei
        \end{itemize}
        
\end{enumerate}


\newpage


\subsection{Teilversuch 3: Duane-Huntsches Verschiebungsgesetz}
\begin{enumerate}[label = (\Roman*)]
    \item Ziel: ...
    
    \item Versuchsmethode: ...
    
    \item Versuchsskizze:
    
        \begin{figure}[H]
        \centering
        \includegraphics[width=0.7\linewidth]{Bild}
        \caption{Versuchsskizze Teilversuch 3}
        \end{figure}

    \item Planung der Durchführung
        \begin{itemize}
            \item eins
            \item zwei
        \end{itemize}

    \item Vorüberlegungen zur Durchführung \& Auswertung
        \begin{itemize}
            \item eins
            \item zwei
        \end{itemize}
    
\end{enumerate}


\newpage


\subsection{Teilversuch 4: Röntgenfluoreszenzanalyse}
\begin{enumerate}[label = (\Roman*)]
    \item Ziel: ...
    
    \item Versuchsmethode: ...
    
    \item Versuchsskizze:
    
        \begin{figure}[H]
        \centering
        \includegraphics[width=0.7\linewidth]{Bild}
        \caption{Versuchsskizze Teilversuch 4}
        \end{figure}

    \item Planung der Durchführung
        \begin{itemize}
            \item eins
            \item zwei
        \end{itemize}

    \item Vorüberlegungen zur Durchführung \& Auswertung
        \begin{itemize}
            \item eins
            \item zwei
        \end{itemize}
    
\end{enumerate}


\newpage


\subsection{Teilversuch 5: Identifiaktion einer unbekannten Probe}
\begin{enumerate}[label = (\Roman*)]
    \item Ziel: ...
    
    \item Versuchsmethode: ...
    
    \item Versuchsskizze:
    
        \begin{figure}[H]
        \centering
        \includegraphics[width=0.7\linewidth]{Bild}
        \caption{Versuchsskizze Teilversuch 5}
        \end{figure}

    \item Planung der Durchführung
        \begin{itemize}
            \item eins
            \item zwei
        \end{itemize}

    \item Vorüberlegungen zur Durchführung \& Auswertung
        \begin{itemize}
            \item eins
            \item zwei
        \end{itemize}
    
\end{enumerate}

\newpage

\section{Versuchsprotokoll}

Auf den folgenden Seiten befindet sich das eingescannte Versuchsprotokoll. Alle Daten wurden selbst gemessen. Sofern fremde Hilfe benutzt wurde, wurde sie klar gekennzeichnet.

Messunsicherheiten wurden angegeben und folgend in der Auswertung verwendet. Alle weiteren Rechnungen und Analysen finden in der Versuchsasuwertung statt.

\includepdf[pages={...}, pagecommand={\thispagestyle{scrheadings}}, frame=true]{[Name.pdf]]}

\newpage

\section{Auswertung}

    \subsection{Teilversuch 1: Bragg-Reflexion von Röntgenstrahlung des Molybdän an einem NaCl-Eiskristall}

    \newpage

    \subsection{Teilversuch 2: Energiespektrum einer Röntgenröhre in Abhängigkeit der Spannung}
    
    \newpage

    \subsection{Teilversuch 3: Duane-Huntsches Verschiebungsgesetz}

    \newpage

    \subsection{Teilversuch 4: Röntgenfluoreszenzanalyse}

    \newpage

    \subsection{Teilversuch 5: Identifiaktion einer unbekannten Probe}

    \newpage

\newpage

\section{Anmerkung: Graphische Auswertung und Fehlerfortpflanzung mit Python-Code}

Alle Berechnungen inkl. Fehlerbestimmung wurden mit einem selbstgeschriebenen
Python-Skript durchgeführt, um uns die Arbeit zu erleichtern und Fehler zu
vermeiden. Alle Ergebnisse, die auf diese Weise zustande gekommen sind,
sind entsprechend mit einem \colorbox{codebg}{blauen Hintergrund} gekennzeichnet;
s. folgendes Beispiel:
\[
    F \wideeq ma \wideeq \result{20}{kg} \cdot 9,81 \, \frac{\text m}{\text s^2}
    \wideeq \result{(19,62 \pm 0,5)}{N} \coderef{tv1}
\]
Dies soll bedeuten, dass die Berechnung des Wertes und der Unsicherheit von der
Python-Funktion namens \verb|tv1| durchgeführt wird.
Die Unsicherheit wird mithilfe der Gauß'schen Fehlerfortpflanzung berechnet.
Außerdem wird das Python-Package \texttt{matplotlib} zum Erstellen
von Graphen verwendet.

Der verwendete Code ist sowohl auf GitHub verfügbar (\githuburl) als auch auf den
folgenden Seiten zu finden und kann mit dem Befehl \texttt{python Main.py}
ausgeführt werden. Für eine genauere Beschreibung des Codes siehe die README-Datei
auf GitHub sowie die Kommentare im Code.
(Manche Sonderzeichen im Code (ä, ö, ü, $\Delta$, etc.) werden von \LaTeX nicht
richtig erkannt, deswegen kann der Code auf den nachfolgenden Seiten an einigen
Stellen unvollständig erscheinen. Auf GitHub wird aber alles richtig angezeigt.)

\newpage


\verb|Main.py|:
\lstinputlisting[language=Python]{Code/Main.py}
\newpage

\verb|Expressions.py|:
\lstinputlisting[language=Python]{Code/Expressions.py}
\newpage

Output:
\lstinputlisting{Code/Output.txt}

\end{document}